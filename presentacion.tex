\documentclass[spanish]{beamer}

%Language symbols
\usepackage[spanish]{babel}
\selectlanguage{spanish}
\usepackage[utf8]{inputenc}

% Code

\usepackage{listings,textcomp}
\lstset{
  breakatwhitespace,
  language=Python,
  columns=fullflexible,
  keepspaces,
  breaklines,
  tabsize=2, 
  showstringspaces=false,
  extendedchars=true,
  basicstyle=\fontfamily{pcr}\selectfont\scriptsize,
  keywordstyle=\color{orange},
  upquote=true,
  literate={-}{-}1}

%Theme
\usetheme{metropolis}

%Title
\title{Iteradores y Generadores}
\date{\today}
\author{Yábir G. Benchakhtir}
\institute{LibreIM}

%Document
\begin{document}
  \maketitle

  \begin{frame}\frametitle{Contenido}
    \tableofcontents
  \end{frame} 
  \section{Iteradores}
  \begin{frame}{¿Qué es un iterador en python?}
    \begin{definition}
      Un \textbf{objeto} representando un flujo de datos.
    \end{definition}

    Tiene las siguientes propiedades:

    \begin{itemize}
    \item Reiteradas llamadas a \textit{\_\_next\_\_()} devuelve elementos del iterador.
    \item Devuelve \textbf{StopIteration} cuando no hay más elementos que devolver.
    \item Necesitan un método \textit{\_\_iter\_\_()} que devuelve el
      propio objeto. Esto hace a todo iterador iterable.
    \end{itemize}
  \end{frame}

  \begin{frame}{Iterables en Python}
    \begin{definition}
      Un objeto capaz capaz de devolver uno de sus miembros cada vez.
    \end{definition}

    Algunos ejemplos de iterables son los tipos \textit{list},
    \textit{str}, \textit{tuple} y algunos no secuenciales como los
    diccionarios o los archivos.
  \end{frame}

  \begin{frame}
    Los objetos iterables se pueden usar en con todas las estructuras
    que usen secuencias. Un bucle \textit{for} es el más tipico pero
    también se puede usar con \textit{map}, \textit{zip}...
  \end{frame}

  \begin{frame}[fragile]
    
    \begin{lstlisting}[language=Python,keywordstyle=\bf,stringstyle=\it]
      Python 3.6.3 (default, Oct  3 2017, 21:45:48) 
      [GCC 7.2.0] on linux
      >>> lst = [0,1,2,3,4,5]
      >>> lst.__iter__
      <method-wrapper '__iter__' of list object at 0x7fba41c74b88>
      >>> iter(lst)
      <list_iterator object at 0x7fba41c7dbe0>
      >>> lstIterator = iter(lst)
      >>> lstIterator.__next__()
      0
      >>> next(lstIterator)
      1
      
    \end{lstlisting}
    
  \end{frame}

  \begin{frame}[fragile]{Iterando}
    Para iterar sobre un \textit{iterador} podemos usar un bucle \textbf{for}.

    \begin{lstlisting}
      >>> lst = ["Python", "Ruby", "JS", "Haskell", "Go"]
      >>> for lang in lst:
      ...     print(lang)
      ... 
      Python
      Ruby
      JS
      Haskell
      Go
      >>> 
    \end{lstlisting}
  \end{frame}

  \begin{frame}[fragile]
    Lo que nunca hay que hacer es:

    \begin{lstlisting}
      >>> for i in range(len(lst)):
      ...     print(lst[i])
      ... 
      Python
      Ruby
      JS
      Haskell
      Go

    \end{lstlisting}

    Algo mejor en caso de necesitar los índices sería:

    \begin{lstlisting}
      >>> for pos, elem in enumerate(lst):
      ...     print(pos, " ", elem)
      ... 
      0   Python
      1   Ruby
      2   JS
      3   Haskell
      4   Go
    \end{lstlisting}
  \end{frame}

  \section{Generadores}
  \begin{frame}{Definición}
    Un generador es una función que usa \textbf{yield} para producir
    una serie de valores. La función devuelve un generador iterador.
  \end{frame}

  \begin{frame}[fragile]
    Un primer ejemplo de generador
    \begin{lstlisting}
      >>> def count(start,num):
      ...     i = start
      ...     while i <= num:
      ...         yield i
      ...         i += 1
      ... 
      >>> for i in count(4,9):
      ...     print(i)
      ... 
      4
      5
      6
      7
      8
      9

    \end{lstlisting}
  \end{frame}


  \begin{frame}[fragile]
    También podemos crear generadores con la sintaxis para crear listas por comprehesion.
    \begin{lstlisting}
      >>> gen = (x for x in range(10))
      >>> next(gen)
      0
      >>> next(gen)
      1
      >>> next(gen)
      2
      >>> gen = (x for x in range(2))
      >>> next(gen)
      0
      >>> next(gen)
      1
      >>> next(gen)
      Traceback (most recent call last):
      File "<stdin>", line 1, in <module>
      StopIteration

    \end{lstlisting}
  \end{frame}

  \section{Itertools}
  \begin{frame}[fragile]{Count}

    Sintaxis:

    \begin{lstlisting}
      itertools.repeat(object[, times])
    \end{lstlisting}

    Ejemplo:

    \begin{lstlisting}
      >>> from itertools import repeat
      >>> a = repeat(2)
      >>> next(a)
      2
      >>> next(a)
      2
      >>> next(a)
      2
      >>> next(a)
      2
      >>> from itertools import repeat
      >>> b = repeat(3,2)
      >>> next(b)
      3
      >>> next(b)
      3
      >>> next(b)
      Traceback (most recent call last):
      File "<stdin>", line 1, in <module>
      StopIteration

    \end{lstlisting}
    
  \end{frame}

  \begin{frame}[fragile]{¿Cuándo es útil?}

    Si queremos repetir algo n veces.

    \begin{lstlisting}
      for _ in itertools.repeat(None, n):
          repetir()
    \end{lstlisting}

    es más rápido que

    \begin{lstlisting}
      for i in range(n):
          repetir()
    \end{lstlisting}
    
  \end{frame}

  \begin{frame}[fragile]

    También cuando trabajamos con \textit{map} y \textit{zip}

    \begin{lstlisting}
      >>> list(map(pow, range(10), repeat(2))) 
      [0, 1, 4, 9, 16, 25, 36, 49, 64, 81]
      >>> list(map(pow, range(10), 2)) 
      Traceback (most recent call last):
      File "<stdin>", line 1, in <module>
      TypeError: 'int' object is not iterable
    \end{lstlisting}
    
  \end{frame}

  \subsection{Filtrar datos}
  \begin{frame}[fragile]{filter}

    Sintaxis:

    \begin{lstlisting}
      itertools.filterfalse(predicate, iterable)
    \end{lstlisting}

    Funcionamiento:

    Devuelve un iterador de elementos del iterable para los que el
    predicado se evalua como falso.
    
  \end{frame}

  \begin{frame}[fragile]{while}

    Sintaxis:
    
    \begin{lstlisting}
      itertools.takewhile(predicate, iterable)
      itertools.dropwhile(predicate, iterable)
    \end{lstlisting}

    Funcionamiento:

    \begin{itemize}
    \item \textit{takewhile} devuelve los elementos del iterable hasta
      que el predicado se evalua como falso para algún elemento.
    \item \textit{dropwhile} ignora los elementos del iterable hasta que el
    predicado se evalua como falso para algún elemento.
    \end{itemize}

  \end{frame}

  \begin{frame}[fragile]

    Sintaxis:
    
    \begin{lstlisting}
      itertools.islice(iterable, stop)
      itertools.islice(iterable, start, stop[, step])
    \end{lstlisting}

    Funcionamiento:

    Toma una porción finita de un generador.
    
  \end{frame}

  \subsection{Maps}

  \begin{frame}[fragile]{accumulate}
    Sintaxis:

    \begin{lstlisting}
      itertools.accumulate(iterable[, func])
    \end{lstlisting}

    Funcionamiento:

    Devuelve un iterador resultado de aplicar una operación binaria de
    manera acumulada. Por defecto usa \textit{operator.sum}
    
  \end{frame}

  \begin{frame}[fragile]{Ejemplo de uso}

    
    
    \begin{lstlisting}
      >>> s = [5,4,2,8,7,6,3,0,9,1]
      >>> list(accumulate(s))
      [5, 9, 11, 19, 26, 32, 35, 35, 44, 45]
      >>> list(accumulate(s,min))
      [5, 4, 2, 2, 2, 2, 2, 0, 0, 0]
      >>> from operator import mul
      >>> list(accumulate(s,mul))
      [5, 20, 40, 320, 2240, 13440, 40320, 0, 0, 0]
    \end{lstlisting}
    
  \end{frame}

  \begin{frame}[fragile]{Ejemplo}
    \begin{lstlisting}
      iterate(f, x) -> x, f(x), f(f(x)), ...
    \end{lstlisting}

    \begin{lstlisting}
      def iterate(f, x):
          yield x
          yield from iterate(f, f(x))
    \end{lstlisting}
        
  \end{frame}


  \begin{frame}[fragile]

    Una versión que no se quedará sin memoria:

    \begin{lstlisting}
      def iterate(f, x):
          while True:
              yield x
              x = f(x)
    \end{lstlisting}
    
  \end{frame}

  \begin{frame}[fragile]

    La versión:
    
    \begin{lstlisting}
      def iterate(f, x):
          return accumulate(repeat(x), lambda fx, _: f(fx))
    \end{lstlisting}

    Ejemplo de \href{https://github.com/joelgrus/stupid-itertools-tricks-pydata}{@joelgrus}.
    
  \end{frame}

  \begin{frame}[fragile]{starmap}

    Sintaxis:

    \begin{lstlisting}
      itertools.starmap(function, iterable)
    \end{lstlisting}

    Funcionamiento:

    Mismo funcionamiento que map pero los argumentos han sido agrupados de manera previa.
    
    
    \begin{lstlisting}
      >>> list(starmap(mul, enumerate("Dark Souls",1)))
      ['D', 'aa', 'rrr', 'kkkk', '     ', 'SSSSSS', 'ooooooo', 'uuuuuuuu', 'lllllllll', 'ssssssssss']
    \end{lstlisting}

    
  \end{frame}

  \subsection{Combinar iteradores}
  \begin{frame}[fragile]{Algunas funciones interesantes son}

    \begin{lstlisting}
      itertools.chain(*iterables)
      itertools.product(*iterables, repeat=1)
      zip(*iterables)
      itertools.groupby(iterable, key=None)
      itertools.permutations(iterable, r=None)
    \end{lstlisting}
    
  \end{frame}

  \section{Extras}

  \begin{frame}[fragile]{Algunas funciones extra}

    \begin{lstlisting}
      >>> all([1,2,3])
      True
      >>> all([1,2,0])
      False
      >>> any([1,2,0])
      True
      >>> any([1,2,3])
      True
      >>> sum([1,2,3])
      6
      >>> from functools import reduce
      >>> from operator import mul
      >>> reduce(mul,[1,2,3,4])
      24

    \end{lstlisting}
    
  \end{frame}
  
  \section{Referencias}
  \begin{frame}\frametitle<presentacion>{Referencias}
    \begin{thebibliography}{10}
    \bibitem[Python Glossary]{https://docs.python.org/3.6/glossary.html}
      \newblock{\href{https://docs.python.org/3.6/glossary.html}{Glosario de Python}}

    \bibitem[Itertools]{https://docs.python.org/3/library/itertools.html}
      \newblock{\href{https://docs.python.org/3/library/itertools.html}{Referencia de Itertools}}
    \end{thebibliography}
  \end{frame}
\end{document}