\documentclass[spanish]{beamer}

%Language symbols
\usepackage[spanish]{babel}
\selectlanguage{spanish}
\usepackage[utf8]{inputenc}

% Code

\usepackage{listings,textcomp}
\lstset{
  breakatwhitespace,
  language=Python,
  columns=fullflexible,
  keepspaces,
  breaklines,
  tabsize=2, 
  showstringspaces=false,
  extendedchars=true,
  basicstyle=\fontfamily{pcr}\selectfont\scriptsize,
  keywordstyle=\color{orange},
  upquote=true,
  literate={-}{-}1}

%Theme
\usetheme{metropolis}

%Title
\title{Iteradores y Generadores}
\date{\today}
\author{Yábir G. Benchakhtir}
\institute{LibreIM}

%Document
\begin{document}
  \maketitle

  \begin{frame}\frametitle{Contenido}
    \tableofcontents
  \end{frame} 
  \section{¿Iteradores?}
  \subsection{Conociendo las herramientas}
  \begin{frame}{¿Qué es un iterador en python?}
    \begin{definition}
      Un \textbf{objeto} representando un flujo de datos.
    \end{definition}

    Tiene las siguientes propiedades:

    \begin{itemize}
    \item Reiteradas llamadas a \textit{\_\_next\_\_()} devuelve elementos del iterador.
    \item Devuelve \textbf{StopIteration} cuando no hay más elementos que devolver.
    \item Necesitan un método \textit{\_\_iter\_\_()} que devuelve el
      propio objeto. Esto hace a todo iterador iterable.
    \end{itemize}
  \end{frame}

  \begin{frame}{Iterables en Python}
    \begin{definition}
      Un objeto capaz capaz de devolver uno de sus miembros cada vez.
    \end{definition}

    Algunos ejemplos de iterables son los tipos \textit{list},
    \textit{str}, \textit{tuple} y algunos no secuenciales como los
    diccionarios o los archivos.
  \end{frame}

  \begin{frame}
    Los objetos iterables se pueden usar en con todas las estructuras
    que usen secuencias. Un bucle \textit{for} es el más tipico pero
    también se puede usar con \textit{map}, \textit{zip}...
  \end{frame}

  \begin{frame}[fragile]
    
    \begin{lstlisting}[language=Python,keywordstyle=\bf,stringstyle=\it]
      Python 3.6.3 (default, Oct  3 2017, 21:45:48) 
      [GCC 7.2.0] on linux
      >>> lst = [0,1,2,3,4,5]
      >>> lst.__iter__
      <method-wrapper '__iter__' of list object at 0x7fba41c74b88>
      >>> iter(lst)
      <list_iterator object at 0x7fba41c7dbe0>
      >>> lstIterator = iter(lst)
      >>> lstIterator.__next__()
      0
      >>> next(lstIterator)
      1
      
    \end{lstlisting}
    
  \end{frame}

  \begin{frame}{Iterando}
    
  \end{frame}
  
  \section{Referencias}

  \begin{frame}\frametitle<presentacion>{Referencias}
    \begin{thebibliography}{10}
    \bibitem[Python Glossary]{https://docs.python.org/3.6/glossary.html}
      \newblock{\href{https://docs.python.org/3.6/glossary.html}{Glosario de Python}}
    \end{thebibliography}
  \end{frame}
\end{document}